\documentclass{article}
\usepackage[margin=1.8in]{geometry}
\usepackage[utf8]{inputenc}
\usepackage{bibentry}
\nobibliography*


\begin{document}

\bibentry{teixeira2018vi}: argued that a collaborative relative pose estimation
is required to enable MAV flights close to structures that may have
insufficient texture near or on the target. The proposed method uses a master
and slave architecture, where the slave MAV is positioned close to the target
structure, and the master MAV acting as a reference to the slave is positioned
further away. The master is retrofitted with a known constellation of LED
markers in order for the slave MAV to estimate the relative pose.

\bibentry{morrison2016moarslam} A collaborative SLAM system is the work of ,
where the system is designed to work with multiple hand-held devices equipped
with either a monocular camera and IMU or stereo camera pair, along with a
mobile phone or autonomous robot for network and computing capabilities. Each
agent performs full SLAM on board, and a centralized server is used for storing
and sharing maps between agents.  However, this system exhibits few other
collaborative features, for one it does not perform a global optimization of
the maps created by different agents, nor are the map optimization work load
shared amongst other agents or server.

\bibentry{deutsch2016framework} a system that can combine different pose graph
based SLAM systems running on the agents is presented. Additionally, the
centralized server informs agents about updates in their respective pose
graphs. However, the global map and sub-maps of other agents are not shared
amongst the agents.

\bibentry{cieslewski2017rapid}: A rapid exploration method for a MAV in an
unknown environment is devised. The core contribution of this work lies in
maintaining a constant velocity while exploring an unknown environment, thereby
reducing the amount of wasted energy hovering in place. During outdoor
experiments, however, they experienced frequent visual odometry failures due to
lighting conditions and sudden quadrotor attitude changes.

\bibentry{palazzolo2017information}: proposed an information-driven path
planning algorithm for a MAV through an unknown environment. The MAV is assumed
to be carrying a forward-facing camera to perform object mapping, and has a 3D
bounding box around the target object of interest as a prior. This method is
validated in simulation only.

\bibentry{zou2013coslam}: CoSLAM obtains multiple monocular camera streams as
input, and groups cameras using a place recognition module. The disadvantage of
this system are, first, all cameras need to be synchronized.  Secondly, to
initialize the SLAM systems all cameras must observe the same scene. Lastly,
the processing of all data requires a graphics processing unit (GPU) to run at
real time.

\bibentry{schmuck2018ccm}: Collaborative centralized multi-agent SLAM
(CCM-SLAM) The system was designed to fuse data streams from multiple monocular
cameras on board multiple MAVs. Each MAV runs a visual odometry system for
autonomy, and offloads the computationally intensive tasks back to the central
server. All information between the MAVs and the central server are shared, for
example if an agent visits an area of the environment previously visited by
another agent, previous experiences are retrieved from the server. Finally, the
server detects and removes redundant data without compromising the robustness
of the estimation.

\bibentry{cunningham2013ddf}: is an example of decentralized SLAM system, where an
anti-factor is introduced in order to remove and replace redundant information
in a graph SLAM problem.

\bibentry{cieslewski2017efficient}: proposes a method that requires a similar
amount of data exchange between agents as a centralized approach without
precluding any matches. This is achieved by pre-assigning visual bag-of-words
vocabulary to different robots, and obtaining a candidate selection to choose
which robot to send the full query.  Both works were validated in simulation
only.

\bibentry{richardson2013iros}: AprilCal guides users to collect a high-quality
camera images inorder to perform camera intrinsics calibration for a monocular
camera.

\bibentry{Rebello2017}: Calibrates the extrinsics between a
stereo camera pair, where one camera is rigidly mounted and the other camera is
actuated by a gimbal. The key contribution of this work is the next-best-view
selection and using it to reduce the total amount of image data for a good
extrinsic calibration.

\bibentry{Bajcsy1988} Actuating a camera intentionally is also known as
active vision. Active vision is by no means new, one of the earliest work can
be traced back to \cite{Bajcsy1988} where active vision is defined as a
problem of an intelligent data acquisition process.

\bibentry{achtelik2011collaborative} demonstrated that two MAVs equipped with
onboard monocular cameras and IMUs can form a flexible stereo rig. This was
achieved by performing feature correspondence in the overlapping field of view
between the MAVs to estimate the relative pose.

\bibentry{zou2013coslam} multiple handheld cameras, but cameras are all
synchronized making it impractical for robotic applications, where the input of
each camera should be computed asynchronoously in order to cope with missing
data and delays. Additionally, it is assumed that all cameras observe the same
scene at the start.

\bibentry{cunningham2013ddf} presents a fully decentralized SLAM system where
each robot maintains a consistent augmented local map that combines local and
neighbourhood information, but the system has only been validated in
simulation.

\bibentry{morrison2016moarslam} designed a collaborative SLAM system to be used
with hand-held devices. The target of this system is to enable multi-device
mapping. They perform full SLAM on each agent, the server is used as a central
memory for storing and sharing the agents' maps. This architecture of this
system does not make use of many advantages a collaborative system offers,
since the expensive optimization algorithms are still run on the agent. The
server detects correspondences between maps, but does not run a global
optimization.

\bibentry{deutsch2016framework} presents a system that can be run on a server
and is able to combine different SLAM systems in a collaborative framework. The
SLAM systems that can be combined in this approach, however, are restricted to
such that operate on a pose graph, where each keyframe has an image and its
absolute scale linked to it. This excludes, for instance, all pure monocular
SLAM systems. Furthermore, as in \cite{morrison2016moarslam}, a full SLAM
system runs on the agent. The server informs an agent about updates in its pose
graph. The global map on the server and the sub-maps of other agents are not
visible to the agents.


% BIBLIOGRAPHY
\bibliographystyle{ieeetr}
\bibliography{index}

\end{document}
