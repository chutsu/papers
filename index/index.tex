\documentclass{article}

\begin{document}

% COMPUTER VISION
\section{Computer Vision}



% CALIBRATION
\section{Calibration}



% MAV MODELLING AND CONTROL
\section{MAV MODELLING AND CONTROL}



% MAV LANDING
\section{MAV Landing}



% SLAM
\section{SLAM}



% VISUAL-INERTIAL-ODOMETRY
\section{Visual-Inertial Odometry}



% VISUAL-ODOMETRY
\section{Visual Odometry}



% MULTI-AGENT
\section{Multi-Agent}

\cite{achtelik2011collaborative} demonstrated that two MAVs equipped with
onboard monocular cameras and IMUs can form a flexible stereo rig. This was
achieved by performing feature correspondence in the overlapping field of view
between the MAVs to estimate the relative pose.

\cite{zou2013coslam} multiple handheld cameras, but cameras are all
synchronized making it impractical for robotic applications, where the input of
each camera should be computed asynchronoously in order to cope with missing
data and delays. Additionally, it is assumed that all cameras observe the same
scene at the start.

\cite{cunningham2013ddf} presents a fully decentralized SLAM system where
each robot maintains a consistent augmented local map that combines local and
neighbourhood information, but the system has only been validated in
simulation.

\cite{morrison2016moarslam} designed a collaborative SLAM system to be used
with hand-held devices. The target of this system is to enable multi-device
mapping. They perform full SLAM on each agent, the server is used as a central
memory for storing and sharing the agents' maps. This architecture of this
system does not make use of many advantages a collaborative system offers,
since the expensive optimization algorithms are still run on the agent. The
server detects correspondences between maps, but does not run a global
optimization.

\cite{deutsch2016framework} presents a system that can be run on a server and
is able to combine different SLAM systems in a collaborative framework. The
SLAM systems that can be combined in this approach, however, are restricted to
such that operate on a pose graph, where each keyframe has an image and its
absolute scale linked to it. This excludes, for instance, all pure monocular
SLAM systems. Furthermore, as in \cite{morrison2016moarslam}, a full SLAM
system runs on the agent. The server informs an agent about updates in its pose
graph. The global map on the server and the sub-maps of other agents are not
visible to the agents.

\end{document}
